\chapter{Related Work}
\label{related works}

\section{Performance of  An Interprocessor CNOT Gate}
Both the execution time and required amount of resource for inter-node communication are important because it significantly affects the total execution time and the whole architecture of a distributed quantum computing system. 

 The work \cite{gateteleportation} proved that one bit of classical communication in each direction and one bell pair are sufficient for the non-local CNOT gate.  It also proposes the optimal implementation of a non-local CNOT in terms of communication overhead and the number of required quantum gates. 
 
 The work \cite{arithmetic} compared the performance between "telegate" and "teledata" approach in terms of various number of data qubits, communication qubits, and its network topology.  It presents the fact that the teledata approach is faster than the telegate approach, and that decomposition of a quantum gate will improve the performance.  It also shows that each node should have a few logical qubits and two communication qubits.

\section{Minimization of the Number of Interprocessor Communication}

Dividing the given quantum circuit into several fragments one of the main jobs for a distributed quantum compiler.  In the process of partitioning the given quantum circuit, the distributed quantum compiler has to minimize the number of inter-processor communication in order to reduce the delay in the entire circuit execution.  

Severals algorithm for finding the quantum circuit partition with minimum number of inter-processor communications have been proposed and these algorithms are based on exhaustive search  \cite{exhaustivesearch}, graph partitioning \cite{Kernighahn-Lin}, hypergraph partitioning \cite{hypergraph}, genetic algorithm \cite{genetic}, dynamic programming \cite{dynamic}, window-based quantum circuit partitioning \cite{WQCP} , and connectivity matrix of the quantum circuit \cite{matrix}.

\section{Quantum Compiler For Distributed Quantum Computing}

Ferrari \textit{et al.}  \cite{distributedquantumcompiler} discussed the design of a general-purpose, efficient, and effective distributed quantum computer. General purpose means no assumption about the given quantum circuit. Efficient means polynomial time complexity that grows polynomially with the number of qubits and linearly with the circuit depth.  Effective assures a polynomial worst-case overhead in terms both depth of the compiled circuit, the number of entanglement generation.

 This study also derived the analytical upper bound of the circuit depth both for the entanglement-based non-local operation and the data-qubit-swapping-based non-local operation.
 
 The depth overhead of the entanglement-swapping-based strategy would be at most 
 
  \begin{equation}
\frac{n}{2} \operatorname{d_{es}}
  \end{equation}
  
 \begin{equation}
  \operatorname{d_{es}} =  \operatorname{c_{le}} +  \operatorname{c_{bsm}} +  \operatorname{c_{cx}}
  \end{equation}.
 
  On the other hand, the depth overhead of the data-qubit-swapping strategy is
  
   \begin{equation} 
  \frac{n}{4} \operatorname{d_{qs}} +  \operatorname{d'_{qs}}
   \end{equation} 
   
   \begin{equation} 
    \operatorname{d_{qs}}= 3( \operatorname{c_{le}} +  \operatorname{c_{bsm}} +  \operatorname{c_{cx}})
    \end{equation}
    
  \begin{equation}
    \operatorname{d'_{qs}} =  \operatorname{c_{le}} + \operatorname{ c_{cx}}
   \end{equation}.
  
  $n$ is the number of qubits, $c_{le}$ is the number of layers required to perform the link entanglement, $c_{bms}$ is that to perform entanglement swapping, $c_{cx}$ is that to perform remote CNOTs.
  This study mentions that parameters $c_{le}$, $c_{bms}$, $c_{cx}$ heavily depends on the underlying hardware architecture.
  
  It also compare the performance of both strategies with the previous work \cite{hypergraph}.  It experimentally demonstrated that the entanglement-swapping-based strategy requires less number of layers for link generation, and the data-qubit-swapping-based strategy requires less circuit depth, on the worst network topology (the linear topology with one qubit on an each processor).

\newpage

\section{Toward Experimental Realization of Distributed Quantum Computing} 

Jiang \textit{et al.}\cite{jiang} proposed a method to perform distributed quantum computing even by using 5-qubit noisy processors and noisy quantum links. 

Nickerson \textit{et al.}\cite{nickerson} proposed a protocol to perform distributed quantum computing which utilizes topological codes to perform purification in the processor and also states that this protocol works if the error rate of quantum links is less than 10\% and that of quantum processor for initialization, state manipulation, and measurement is less than 0.82\% 

Oi \textit{et al.}\cite{oi} proposed an architecture for distributed, fault-tolerant quantum computing that enables scalable that achieves scalable quantum error correction. 

Daiss \textit{et al.}\cite{daiss} experimentally achieved a remote quantum gate and successfully implemented four Bell states. 

Kim \textit{et al.}\cite{kim} proposed an architecture for a large-scale quantum computing that combines ion-trap and optical technologies. 

LaRacuente \textit{et al.}\cite{laracuente} proposed an architecture of a superconducting chiplet that connects the physical processor and microwave links. 

Van Meter \textit{et al.}\cite{van} discusses required technology stack for distributed quantum computing in terms of various hardware technologies and software that manipulates these technologies.




%%% Local Variables:
%%% mode: japanese-latex
%%% TeX-master: "./thesis"
%%% End:
