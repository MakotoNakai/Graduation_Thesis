Abstract of Bachelor's Thesis - Academic Year 2021
\begin{center}
\begin{large}
\begin{tabular}{|p{0.97\linewidth}|}
    \hline
      \etitle \\
    \hline
\end{tabular}
\end{large}
\end{center}

~ \\
 Quantum computers are theoretically capable of solving some intractable problems for classical computers including the factoring problem and the digital quantum simulations in a polynomial time, and large-scale quantum computing is the key to perform computation that even a supercomputer cannot handle.   Two approaches for large-scale quantum computing have been proposed. One is to build a single large quantum processor, and the other is to perform quantum computing over more than one quantum processors that are connected via communication links.  The later approach is considered as more practical because it requires less number of qubits and error rate, which are the two most challenging aspects of building a hardware for large-scale quantum processor.
 
 However, only few works have investigated the procedure to perform distributed quantum computing in the real world, especially how to convert a user program which includes the quantum circuit to the executable form onto distributed quantum computers.
 
 This work proposes the heuristic optimization algorithm for qubit allocation for distributed quantum computing which aim to shorten the total execution time of the given quantum circuit, and demonstrated that the qubit allocation by this algorithm achieve reduction of the total execution time compared to the random qubit allocation by numerical simulation.
~ \\
Keywords : \\
\underline{1. Quantum computing},
\underline{2. Distributed system},
\underline{3. Task allocation},
\underline{4. Simulated Annealing}
\begin{flushright}
\edept \\
\eauthor
\end{flushright}
