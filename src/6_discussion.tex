\chapter{Discussion}
\label{discussion}

\section{Discussion}
Although, as shown in the last chapter, the case when only the gate execution time is optimized yielded the best result, both the case when gate execution cost is optimized,  and when both costs are optimized reached to the most optimal qubit allocation, which are (1,3) and (0, 2, 4, 5).  The difference between the total execution times between these two cases happened due to the nature of parallel execution in my simulator, which is that each gate is executed in not rigorously sequential order, and also the fact that the randomness caused the delay of communication between two threads, which emulate quantum processors.

More experiments which uses more than two processors with limited network topologies have to be performed in order to validate the effectiveness of allocation method which optimize both costs, rather than either only the gate execution cost, or the communication cost.




%%% Local Variables:
%%% mode: japanese-latex
%%% TeX-master: "./thesis"
%%% End:
