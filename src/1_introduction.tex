\chapter{Introduction}
\label{introduction}

\section{Background}
\label{introduction:background}
 In 1982, the idea of quantum computing was proposed by Richard Feynman \cite{Feynman} with the intention of simulating quantum system whose dimension grows exponentially if the number of particles increase.  In 1990s, quantum computers are proved to be theoretically capable of solving some intractable problems for classical computers including the factoring problem \cite{Shor} and the digital quantum simulations in a polynomial time \cite{Lloyd}, and large-scale quantum computing is the key to perform computation that even a supercomputer cannot handle.  The hardware for quantum computing has been developed using various physical architecture such as superconducting system \cite{superconducting}, ion trap \cite{iontrap}, nvdiamond \cite{nvdiamond}.
 
  Two approaches for large-scale quantum computing have been proposed. One is to build a single large quantum processor \cite{rsa}, and the other is to perform quantum computing over more than one quantum processors that are connected via communication links \cite{grover}.  
 
 However, unlike the previous works about the system for quantum computing on a single quantum processor \cite{qubitallocation, noisy},  only few works have investigated the procedure to perform distributed quantum computing in the real world \cite{qmpi}, especially how to convert an user program which includes the quantum circuit to the executable form onto distributed quantum computers.

\section{Research Contribution}
\label{introduction:research_contribution}
The main contribution of this project is the heuristic optimization algorithm for qubit allocation problem for distributed quantum computing, which determines which qubits and associated quantum gates will be executed on which quantum processor.  This work adopts the total execution time as the optimization criteria, because the algorithm for qubit allocation with emphasis on total fidelity of the output quantum state can be solved by the existing quantum compilation techniques.

\section{Thesis Structure}
\label{introduction:thesis_structure}
This thesis is constructed as follows.

Chapter 2 provides the fundamentals knowledge of quantum information, distributed system, and distributed quantum computing.
In chapter 3 provides the previous researches in the fields of distributed quantum computing that are directly related to this work.
Chapter 4 describes the details of the qubit allocation problem for distributed quantum computing and propose its heuristic solution.
Chapter 5 explains the setting of the experiments and their results.
Chapter 6 discusses the validity of the proposed approach.
Chapter 7 provides a summary of this research and describes the future development of the research.


%%% Local Variables:
%%% mode: japanese-latex
%%% TeX-master: "../thesis"
%%% End:
