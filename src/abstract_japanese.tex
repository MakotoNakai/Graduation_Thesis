卒業論文要旨 - 2021年度 (令和3年度)
\begin{center}
\begin{large}
\begin{tabular}{|M{0.97\linewidth}|}
    \hline
      \title \\
    \hline
\end{tabular}
\end{large}
\end{center}

~ \\

量子コンピュータは従来のコンピュータが現実的な時間内に解けないことが示されている問題の一部のの解より少ない時間計算量で得られることが理論的に示されている。大規模な量子計算を行うアプローチとして大規模かつ単一の量子プロセッサを構築するアプローチと複数の量子プロセッサ上で分散的に行うアプローチの2つが提案されているが、後者の方が量子プロセッサに要求される量子ビット数とエラー率が低いことから、より現実的であると考えられている。
しかし、実際に分散量子計算を行う方法、特に量子回路のプログラム上の各量子ビットをどの量子プロセッサ上で実行するかを決定する方法について議論してされていない。
本研究では量子回路全体の実行時間の短縮を目的とした、量子ビット割り当て問題に対する近似最適化アルゴリズムを提案し、数値シミュレーションで無作為の量子ビット割り当てに比べて、本研究が提案するアルゴリズムで最適化した量子ビット割り当てを採用した方が実行時間が短くなることをシミュレーションで実証した。

~ \\
キーワード:\\
\underline{1. 量子計算},
\underline{2. 分散システム},
\underline{3. 仕事割り当て問題},
\underline{4. 擬似焼きなまし法}
\begin{flushright}
\dept \\
\author
\end{flushright}
