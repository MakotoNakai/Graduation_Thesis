卒業論文要旨 - 2021年度 (令和3年度)
\begin{center}
\begin{large}
\begin{tabular}{|M{0.97\linewidth}|}
    \hline
      \title \\
    \hline
\end{tabular}
\end{large}
\end{center}

~ \\

量子インターネットとは、量子もつれを利用した新しいインターネットであり、現在のインターネットの弱点を補う技術として期待されている。
具体的には、より強固な暗号アルゴリズム、コンセンサス問題の高速解決、分散量子計算による計算能力の向上などが挙げられる。
しかし、実用化はまだまだ遠く、実験段階にある。

実用化する際に必要な要素の一つとしてトラフィックテストがある。
これはネットワークのテストであり、ユーザーが利用する際にプロトコルなどが正しく動作するかどうかを確認するために行われる。
トラフィックテストのためにはトラフィックデータが必要であり、それを用意するには、実データを使うか、もしくは確率的に生成する必要がある。
量子インターネットの場合には実データが存在しないので、データを生成する必要がある。

本プロジェクトでは、量子インターネットにおける「トラフィック」を定義し、現行のインターネットにおいて主流である重力モデルを使ったトラフィックデータ生成を量子インターネットに適用した。
また、これを量子インターネットシミュレータであるQUISPに実装し、テストを行った。
これにより、量子インターネットにおけるトラフィックテストの基盤を示した。

~ \\
キーワード:\\
\underline{1. 量子インターネット},
\underline{2. トラフィック行列},
\underline{3. 重力モデル},
\underline{4. トラフィック生成}
\begin{flushright}
\dept \\
\author
\end{flushright}
